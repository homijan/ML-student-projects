\documentclass[11pt,a4paper]{article}

\usepackage[czech]{babel}  
\usepackage[utf8]{inputenc} 
\usepackage{graphicx} 
\usepackage{color} 
%\usepackage{hyperref}
\usepackage[hyphens]{url}

\usepackage{etoolbox}
\patchcmd{\thebibliography}{\section*{\refname}}{}{}{}

\textwidth155mm
\textheight240mm
\oddsidemargin5mm
\evensidemargin5mm
\topmargin-15mm
\parindent0pt


\begin{document}
\pagestyle{empty}

%\begin{center}
\parbox{0.65\textwidth}{%
{\footnotesize FACULTY OF NUCLEAR SCIENCES \\%[1mm]
               AND PHYSICAL ENGINEERING \\%[1mm]
               DEPARTMENT OF PHYSICS}\\[3mm]
{\bfseries\Large BACHELOR THESIS ASSIGNMENT}\\[3mm]
{\bf Academic year:} 2022/2023
}
%\qquad
\parbox{0.36\textwidth}{%
  \includegraphics[width=0.35\textwidth]{logo_CVUT_EN_Pantone}
}

\vspace*{10ex}

%\noindent
\hspace{-1ex}\begin{tabular}{ll}
{\it Student:}          & Aleksandr Bogdanov\\\\\\
{\it Study programme:}  & Physical Engineering\\\\\\
{\it Specialization:}   & Physics of Plasma and Thermonuclear Fusion\\\\\\
{\it Thesis title:  }   & Aplikace strojového učení při nelokálním hydrodynamickém  \\ 
{\it (in Czech)       } & modelování termojaderné fúze \\\\\\ 
{\it Thesis title:  }   & Machine Learning-Driven Nonlocal Hydrodynamics
                          for Thermonuclear  \\
{\it (in English) }     & Fusion Modeling \\\\\\
{\it Language of the}   & \\
{\it Thesis:}           & English
\end{tabular}

\vspace*{10ex}


{\it Instructions:}

\vspace{3ex}

\begin{enumerate}
\item
  Get acquainted with the state-of-the-art inertial confinement fusion (ICF)
  research and the importance of the physical phenomena of
  transport~\cite{lit:Abu-Shawareb_etal-PRL:22,lit:Casey_etal-PRL:21,lit:Rosen_etal-HEDP:11}.
\item
  Research hydrodynamic models currently used in ICF with the focus on nonlocal
  electron transport~\cite{lit:Holec_Nikl_Weber-PoP:18}.
\item
  Process kinetic modeling data provided by Lawrence Livermore National Laboratory.
\item
  Teach a deep neural network (DNN) to learn the process of nonlocal electron
  transport based on physically motivated loss
  function~\cite{lit:PyToTut,lit:PyToInt,lit:PyToReg}.
\item
  Compare the DNN model with the classical heat flux limiter model used
  in ICF~\cite{lit:Chapman_etal-PoP:21}.
\end{enumerate}



\vspace{2ex}

\newpage

{\it Recommended literature:}
\vspace{1ex}


%\bibliographystyle{plain}
\bibliographystyle{unsrt}
%\bibliographystyle{amsalpha}
%\bibliographystyle{abbrvnat}
%\bibliographystyle{abbrv}
\bibliography{references}

\vspace{4ex}

{\it Name and affiliation of the supervisor:}\\[1mm]
Ing. Milan Holec, Ph.D.\\[1mm]
Lawrence Livermore National Laboratory, CA, USA

\vspace{3ex}

{\it Name and affiliation of the consultant:}\\[1mm]
doc. Ing. Pavel Váchal, Ph.D. \\[1mm]
Department of Physical Electronics, FNSPE CTU in Prague

\vspace{3ex}

%\noindent
\hspace{-1ex}\begin{tabular}{ll}
{\it Date of the assignment:}      & 20.10.2022\\[1mm]
{\it Due date of the thesis:}  & 02.08.2023\\[1mm]
\multicolumn{2}{l}{\it The assignment is valid for two years since the date of the assignment.}
\end{tabular}

\vspace{4ex}

\parbox{0.40\textwidth}{%
\begin{center}
...........................................................\\
  {\it Guarantor of the study programme}
\end{center}
}

\vspace{2ex}

\parbox{0.40\textwidth}{%
\begin{center}
...........................................................\\
  {\it Department head}
\end{center}
}
\qquad\qquad
\parbox{0.40\textwidth}{%
\begin{center}
...........................................................\\
  {\it Dean}
\end{center}
}

\vspace{2ex}

In~Prague on 20.10.2022

\end{document}

