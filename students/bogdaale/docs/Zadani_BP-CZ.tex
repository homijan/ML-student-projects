\documentclass[11pt,a4paper]{article}

\usepackage[czech]{babel}  
\usepackage[utf8]{inputenc} 
\usepackage{graphicx} 
\usepackage{color} 
%\usepackage{hyperref}
\usepackage[hyphens]{url}

\usepackage{etoolbox}
\patchcmd{\thebibliography}{\section*{\refname}}{}{}{}

\textwidth155mm
\textheight240mm
\oddsidemargin5mm
\evensidemargin5mm
\topmargin-15mm
\parindent0pt

%\hyphenation{po\-u\-ží\-va\-ným}

\begin{document}
\pagestyle{empty}

%\begin{center}
\parbox{0.60\textwidth}{%
{\footnotesize FAKULTA JADERNÁ A FYZIKÁLNĚ INŽENÝRSKÁ}\\[1mm]
{\footnotesize KATEDRA FYZIKY}\\[3mm]
{\bfseries\Large ZADÁNÍ BAKALÁŘSKÉ PRÁCE}\\[3mm]
{\bf Akademický rok:} 2022/2023
}
\qquad
\parbox{0.36\textwidth}{%
  \includegraphics[width=0.35\textwidth]{logo_CVUT_CZ_Pantone}
}

\vspace*{10ex}

%\noindent
\hspace{-1ex}\begin{tabular}{ll}
{\it Student:}          & Aleksandr Bogdanov\\\\\\
{\it Studijní program:} & Fyzikální inženýrství\\\\\\
{\it Specializace:}     & Fyzika plazmatu a termojaderné fúze\\\\\\
{\it Název práce:  }    & Aplikace strojového učení při nelokálním hydrodynamickém  \\ 
{\it (česky)       }    & modelování termojaderné fúze \\\\\\ 
{\it Název práce:  }    & Machine Learning-Driven Nonlocal Hydrodynamics
                          for Thermonuclear  \\
{\it (anglicky) }       & Fusion Modeling \\\\\\
{\it Jazyk práce:}      & angličtina
\end{tabular}

\vspace*{10ex}


{\it Pokyny pro vypracování:}

\vspace{3ex}


V~rámci bakalářské práce proveďte následující úkony:
\begin{enumerate}
\item
  Seznamte se se současným stavem a nejnovějšími poznatky z~oblasti
  výzkumu inerciální termojaderné fúze (ICF) a důležitostí fyzikálního jevu tzv.
  transportu~\cite{lit:Abu-Shawareb_etal-PRL:22,lit:Casey_etal-PRL:21,lit:Rosen_etal-HEDP:11}.
\item
  Proveďte rešerši hydrodynamických modelů používaných pro simulace fúzních
  experimentů, zaměřte se především na nelokální elektronový
  transport~\cite{lit:Holec_Nikl_Weber-PoP:18}.
\item
  Zpracujte data získaná z kinetických modelů v Lawrence Livermore National Laboratory.
\item
  Naučte hlubokou neuronovou síť (DNN) učit se proces nelokálního transportu na
  základě správně definované, fyzikálně motivované ztrátové
  funkce~\cite{lit:PyToTut,lit:PyToInt,lit:PyToReg}.
\item
  Porovnejte model neuronové sítě s klasickým modelem limiteru tepelné vodivosti po\-u\-ží\-va\-ným pro ICF~\cite{lit:Chapman_etal-PoP:21}.
\end{enumerate}



\vspace{2ex}

\newpage

{\it Doporučená literatura:}
\vspace{1ex}


%\bibliographystyle{plain}
\bibliographystyle{unsrt}
%\bibliographystyle{amsalpha}
%\bibliographystyle{abbrvnat}
%\bibliographystyle{abbrv}
\bibliography{references}

\vspace{4ex}

{\it Jméno a pracoviště vedoucího bakalářské práce:}\\[1mm]
Ing. Milan Holec, Ph.D.\\[1mm]
Lawrence Livermore National Laboratory, CA, USA

\vspace{3ex}

{\it Jméno a pracoviště konzultanta:}\\[1mm]
doc. Ing. Pavel Váchal, Ph.D. \\[1mm]
Katedra fyzikální elektroniky, FJFI ČVUT v Praze

\vspace{3ex}

%\noindent
\hspace{-1ex}\begin{tabular}{ll}
{\it Datum zadání bakalářské práce:}      & 20.10.2022\\[1mm]
{\it Termín odevzdání bakalářské práce:}  & 02.08.2023\\[1mm]
\multicolumn{2}{l}{\it Doba platnosti zadání je dva roky od data zadání.}
\end{tabular}

\vspace{4ex}

\parbox{0.40\textwidth}{%
\begin{center}
...........................................................\\
  {\it garant studijního programu}
\end{center}
}

\vspace{2ex}

\parbox{0.40\textwidth}{%
\begin{center}
...........................................................\\
  {\it vedoucí katedry}
\end{center}
}
\qquad\qquad
\parbox{0.40\textwidth}{%
\begin{center}
...........................................................\\
  {\it Děkan}
\end{center}
}

\vspace{2ex}

V~Praze dne   20.10.2022

\end{document}



